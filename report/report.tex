\documentclass[a4paper]{scrartcl}

\usepackage[a4paper,top=2.0cm,bottom=1.5cm,left=2.5cm,right=2.0cm,%
headheight=15pt]{geometry}
\addtolength{\footskip}{-6mm}

\usepackage[english,ngerman]{babel}
\usepackage[utf8]{inputenc}
\usepackage{amsmath}
\usepackage{amssymb}
\usepackage{amstex}
\usepackage{color}
\usepackage{hyperref}
\usepackage[pdftex]{graphicx}



\title{Molecular Dynamics Simulation}
\subtitle{Compuational Physics 2016}<++>

\author{Peter Leinweber, Florian Elleringmann}
\hypersetup{pdftitle=Molecular Dynamics Simulation,pdfauthor=Florian Elleringmann \& Peter Leinweber,pdfsubject=comp_phys_2016}
\date{\today}

\begin{document}
\maketitle


\section{Einleitung und Motivation} \label{sec:einl}

Computersimulationen gewinnen in der heutigen Wissenschaftslandschaft durch ihre universelle Einsatzmöglichkeiten immer mehr an Bedeutung. Vorteile ergeben sich durch die prinzipiell leicht Reproduzierbarkeit im Vergleich zu Experimenten; auch können Extrembedingungen, welche nur sehr schwer in Laboren herzustellen sind, betrachtet werden. Einsatz findet die Simulation auch dort, wo strukturell und dynamische Eigenschaften z.B. einer Flüssigkeit dem Experimentator verborgen bleiben und diese berechnet werden müssen.\\
Ein Beispiel für MD-Simulationen ergibt sich aus dem zu Grunde liegenden Projekt. Anhand einer festgelegten Teilchenzahl sollen die Gleichgewichts- und Transporteigenschaften eines Gases simuliert und visualisiert werden, eines klassischen Vielteilchensystems also.\\
Eine weiteres Beispiel für die Einsatzmöglichkeiten der molekulardynamischen Simulation liegt in der Biologie bzw. Pharmakologie. Heutzutage ist bekannt, dass es zu falschen Faltungen von Proteinen kommen und dass diese sog. "misfolds" schwerwiegende Krankheiten auslösen können. Mit Hilfe der MD-Simulationen ist man in der Lage zu verstehen, wie sich solche Proteine falten und auf diesem Wissen aufbauend Medikamente zu entwickeln.

\subsection{This is a subsection}
\label{sec:subsec}
You can refere to Sect.~\ref{sec:sec1} or to a figure
(Fig.~\ref{fig:plot}).

\begin{figure}
\begin{center}
%\includegraphics[]{your_plot.pdf}
\caption{Explain your figure}
\label{fig:plot}
\end{center}
\end{figure} 

You can also insert tables, equations, references and do many more
things. 


Check the manual that is in Moodle.

\end{document}
